%bib=bibtex
\documentclass[a4paper]{article}

\usepackage[utf8]{inputenc}
\usepackage{ifthen,etoolbox}
\usepackage[british]{babel} %languages nicely, with proper date format courtesy of [british]
\usepackage{csquotes} %helps to quotes nicely
\usepackage[hidelinks]{hyperref} %hyperlinks contents and references
\usepackage{graphicx} %required for image manipulation
\usepackage[a4paper, top=25mm,bottom=25mm,left=25mm,right=25mm]{geometry} %allows manipulation of page structure
\usepackage{cite} %BibTeX bibliography manager
\usepackage{amsmath, amssymb} %adds many fancy maths symbols and things
%\usepackage{tikz, pgfplots} %for drawing diagrams and charts
\usepackage{float} %for control over figure positioning
%\usepackage[final]{microtype} %sorts out all typography, line wrapping and spacing for maximum loveliness

\usepackage{ifthen}
\newboolean{uprightparticles}
\setboolean{uprightparticles}{false} %Set true for upright particle symbols
\usepackage{xspace}
\usepackage{upgreek}
\include{lhcb-symbols-def}
\graphicspath{{../../plots/}}

\title{Analysis Note: Measurement of forward $W$ and $Z$ boson cross sections in $pp$ collisions at $\sqrt{5}=\rm{TeV}$}
\author{Laura Cairns}
\date{\today}

\begin{document}
\maketitle


\section{Z Cross Section from 5TeV Experimental Data} \label{results}

The LHCb dataset used in this analysis has been taken at $\sqrt{s} = 5$ TeV. The dataset contains 5840 entries. The workflow of the code used to analyse this data is given in Section \ref{workflow}, with results given in Section \ref{results}, and plot given in Section \ref{Z histograms}.

\subsection{Integrated Luminosity} \label{lumi_val}
Integrated Luminosity $L$ assumed to have percentage uncertainty of 5.0\%.\\$L = 64.6 \pm 3.2 \; {\rm pb^{-1}}$\\

\subsection{Trigger Efficiency} \label{trigger_val}
Trigger Efficiency = $0.9957 \pm 0.0009$\\Trigger Efficiency Relative Uncertainty = $0.00086$\\

\subsection{Z Counts} \label{counts_val}
Counts in Fiducial Region = $3748 \pm 61 \; \rm counts$\\Counts Relative Uncertainty = $0.0163$\\

\subsection{Z Cross Section} \label{xsec_val}
\begin{equation}
\sigma = 58.26 \pm 0.95_{\rm stat} \pm 0.05_{\rm eff} \pm 2.91_{\rm lumi} \; \rm pb.
\end{equation}



\section{Workflow of Code} \label{workflow}

\subsection{Comparison of Experimental Data with Monte Carlo Simulation} \label{sec: Z MC comparison}
In the code \texttt{plot$\_$MC$\_$comparison.cpp}, histograms are plotted using ROOT, comparing a number of variables from the 5 TeV LHCb dataset to a Monte Carlo simulation of the experiment. The Monte Carlo simulation was created in \textit{Pythia} and has 338926 entries, which have been passed through detector reconstruction. Simulation data has been scaled to the experimental data in the histogram plots, with each plotted using 100 bins.
The variables compared are:
\begin{itemize}
  \item Transverse Momentum $p_T$, separately for both positive and negative muons
  \item Pseudorapidity $\eta$, separately for both positive and negative muons
  \item Azimuthal angle $\phi$, separately for both positive and negative muons
  \item Mass of the Z boson $m_Z$
\end{itemize}

Differences can be seen between the experimental and simulated data due to background in the experimental data, and inaccuracies in the simulation. These inaccuracies arise due to the leading order precision of \textit{Pythia} and limits on the precision to which the detector can be described.

See plots in Section \ref{Z histograms}.

\subsection{Luminosity Calculation} \label{sec: Z lumi}
The code \texttt{get$\_$luminosity.py} calculates the integrated luminosity $L$ as the sum of 38 entries, where each entry represents a fraction of the data collected in the 5 TeV run.
The error on luminosity is assumed to be 5$\%$.
Integrated luminosity and its error are output as a json file, with their values given in Section \ref{lumi_val}.
 
\subsection{Trigger Efficiency Calculation} \label{sec: Z trigger efficiency}
The code \texttt{measure$\_$trigger$\_$eff.py} calculates the trigger efficiency, which forms the dominant systematic error in Z cross section calculation. 
Events where the positive muon or negative muon trigger are counted, as well as the events where both muons trigger. 
By requiring that one muon triggers (the tag muon), then there is no bias on the other muon (the probe muon). Hence, the efficiency of the probe muon trigger can be calculated as:
\begin{equation}
{\rm efficiency}_{{\rm probe \;  muon}} = \frac{{\rm number \; of \; events \; where \; both \; muons \; trigger}}{{\rm number \; of \; events \; where \; tag \; muon \; triggers}}.
\end{equation}

The trigger efficiency is then calculated as the efficiency of either muon causing a trigger to activate (TOS).
Trigger efficiency and its relative uncertainty are output as a json file, with the trigger efficiency given in Section \ref{trigger_val}.

\subsection{Cross Section Calculation} \label{sec: Z xsec}
The code \texttt{measure$\_$xsec.py} calculates the integrated cross section of Z $\sigma_Z$.
The number of events in the fiducial region are counted. The fiducial region is comprised of the kinematic cuts:

\begin{itemize}
  \item $60 < m_Z < 120 \; {\rm GeV}$ 
  \item $p_T > 20 \; {\rm GeV}$ for both positive and negative muons
  \item $2.0 < \eta < 4.5$ for both positive and negative muons
\end{itemize}

$\sigma_Z$ is then calculated as
\begin{equation}
\sigma_Z = \frac{{\rm counts}}{L \times {\rm \; trigger \; efficiency}},
\end{equation}
using the previously calculated values of integrated luminosity $L$ and trigger efficiency, read in from their respective codes.

Error analysis on $\sigma_Z$ is carried out using the relative uncertainities on the number of counts, trigger efficiency, and luminosity. Each of these uncertainties is given separately in the result, with the uncertainty on the number of counts representing the statistical error. The cross section is output to a json file, and its value can be seen in Section \ref{xsec_val}, whilst the value of counts can be seen in Section \ref{counts_val}.

\section{Monte Carlo Comparison for 5 TeV Measurement} \label{Z histograms}

\includegraphics[clip, trim = 0.5cm 0cm 1.7cm 1.3cm, width=0.5\textwidth]{Measurement_mup_PT.png}
\includegraphics[clip, trim = 0.5cm 0cm 1.7cm 1.3cm, width=0.5\textwidth]{Measurement_mum_PT.png}
\includegraphics[clip, trim = 0.5cm 0cm 1.7cm 1.3cm, width=0.5\textwidth]{Measurement_mup_ETA.png}
\includegraphics[clip, trim = 0.5cm 0cm 1.7cm 1.3cm, width=0.5\textwidth]{Measurement_mum_ETA.png}
\includegraphics[clip, trim = 0.5cm 0cm 1.7cm 1.3cm, width=0.5\textwidth]{Measurement_mup_PHI.png}
\includegraphics[clip, trim = 0.5cm 0cm 1.7cm 1.3cm, width=0.5\textwidth]{Measurement_mum_PHI.png}
\includegraphics[clip, trim = 0.5cm 0cm 1.7cm 1.3cm, width=0.5\textwidth]{Measurement_Z_M.png}

\section{W Boson Analysis} \label{sec: W boson}

\subsection{Plotting W Data} \label{sec: W plotting}

The $p_T$ distributions of $W \xrightarrow{} \mu \nu_\mu$ decays are plotted for both $W^+$ and $W^-$ in the program \texttt{plot$\_$W$\_$dist.cpp}. 
The $W$ signal is isolated by calculating the isolation; the sum of $p_T$ for all particles surrounding the muon in a cone satisfying
\begin{equation}
    \sqrt{\Delta\eta^2 + \Delta\phi^2} < 0.5.
    \label{eq: isolation}
\end{equation}
The inverse of transverse momentum $p_T^{-1}$ are plotted against this isolation, as shown for both $W^+$ and $W^-$ in the histograms below (Figure \ref{fig: LEGO}. These figures use the LEGO plotting option. An additional plot is provided of the COLZ plotting option for $W^+$ in Figure \ref{fig: COLZ}. 
The $W$ decay signal can be seen as a peak between $p_T^{-1}$ = 0.02 and 0.03 GeV$^{-1}$ at low values of isolation, shown prominently in the COLZ plot. The LEGO plots highlight the charged $\pi/K$ background, identified as the Gaussian peak at high $p_T^{-1}$.


\begin{figure*}[t]
\centering
\includegraphics[clip, trim = 0.3cm 0.3cm 1.7cm 1.3cm,width=0.49\textwidth]{W+_isolated_signal.pdf}
\includegraphics[clip, trim = 0.3cm 0.3cm 1.7cm 1.3cm,width=0.49\textwidth]{W-_isolated_signal.pdf} \hspace{3mm}
\vspace{-8mm}
\caption{\small Histogram plots of inverse transverse momentum $p_T^{-1}$ against the log of the isolation for the $W \xrightarrow{} \mu \nu_\mu$ decay. $W^+$ and $W^-$ are shown left and right respectively Plotted using the LEGO option.}
\label{fig: LEGO}
\end{figure*}

\begin{figure}[!h]
    \centering
    \includegraphics[clip, trim = 0cm 0cm 0.3cm 1.3cm, width=\textwidth]{W+_isolated_signal_COLZ.pdf}
    \vspace{-4mm}
    \caption{\small COLZ plot of inverse transverse momentum $p_T^{-1}$ against the log of the isolation for the $W \xrightarrow{} \mu \nu_\mu$ decay for $W^+$.}
    \label{fig: COLZ}
\end{figure}


\subsection{W Background Analysis} \label{sec: W background}

The code \texttt{W$\_$background.cpp} plots and analyses the various background contributions to the $W^+$ and $W^-$ data.

The files used are:
\begin{itemize}
    \item Data: \texttt{5TeV$\_$2017$\_$32$\_$Down$\_$EW.root}
    \item Simulation of W signal: \texttt{5TeV$\_$2015$\_$24r1$\_$Down$\_$W$\_$Sim09d.root}
    \item Simulation of Z production: \texttt{5TeV$\_$2015$\_$24r1$\_$Down$\_$Z$\_$Sim09d.root}
\end{itemize}

The code includes a number of functions which carry out different parts of the analysis, allowing for $W^+$ and $W^-$ data to be analysed simultaneously. The purposes of the functions are as follows:
\begin{itemize}
    \item \texttt{output$\_$histogram}: Plots a histogram from a TH1F object and outputs as a "png" file, used for testing purposes.
    \item \texttt{make$\_$histogram}: Creates and returns a TH1F histogram from an input decay tree, expression, and cuts.
    \item \texttt{background$\_$fit}: Analyses $\pi/K \xrightarrow{} \mu\nu$ background, fitting the W "SingleTrackNoBias" trees in the data with an exponential, and returning a Monte Carlo template of the fit. A plot of the exponential fit is output as a "png" file.
    \item \texttt{fraction$\_$fitter}: TFractionFitter is carried out on the background contributions, returning the fractions of each in the $W$ data. A plot of this fit is output as a "png file. 
    \item \texttt{produce$\_$fit$\_$model}: Produces a plot of the fit model, output as a "png" file. Each background contribution is scaled to the data by the fraction calculated in \texttt{fraction$\_$fitter}, and the fit model is calculated as the sum of all background contributions.
\end{itemize}
A structure \texttt{fit$\_$fractions} is also introduced to store the fit fractions calculated by the function \texttt{fraction$\_$fitter}. The structure includes the value and error of each fit fraction, along with the string used to label its value in a plot. This forms the output of the function \texttt{fraction$\_$fitter}, and is passed as an input to \texttt{produce$\_$fit$\_$model}.

When creating histogram, the following event selection is carried out, with kinematic cuts carried out on both $W$ and $Z$ histograms, applied to both muons in $Z$.
\begin{itemize}
    \item Kinematic cuts: $p_T >$ 20, 2 $< \eta <$ 4.5
    \item Additional W cuts: $\log_{10}$(Isolation) $<$ 3 (defined in Equation \ref{eq: isolation})
    \item Additional Z cuts: 60 GeV $< m_Z <$ 120 GeV, where $m_Z$ is the invariant mass of $Z$
\end{itemize}

The analysis is carried out in the following steps:
\begin{enumerate}
    \item Make and retrieve background templates. Each is a TH1F histogram of $p_T$ plotted with range 20 to 60 GeV. This is carried out for both $W^+$ and $W^-$, with the number of events in each histogram after event selection shown in Table \ref{tab: W events}.
    \begin{enumerate}
        \item $\pi/K$ background: Function \texttt{background$\_$fit} is run. $p_T$ is plotted for the Wp and Wm ``SingleTrackNoBias" decay trees in the data file, and fit with an exponential. This is plotted, as shown in Figure \ref{fig: W exp back fit}. These decay trees contain high $p_T$ tracks without the requirement for events to be identified as muons (as in the Wp and Wm ``Iso" trees). Hence, these trees contain mostly charged pions $\pi$ and kaons $K$, forming the $\pi/K$ background.
        \newline The slope of the exponential fit is then set to 0.2 which is used to create the background template. This is a TH1F histogram of 10000 Monte Carlo events generated from the fit.
        \item Signal: Function \texttt{make$\_$histogram} is run. $p_T$ is plotted for the Wp and Wm ``Iso" decay trees in the W simulation file, with W cuts. These histograms are statistically limited, as seen in Table \ref{tab: W events}.
        \item Z background: Function \texttt{make$\_$histogram} is run. $p_T$ is plotted for the Wp and Wm ``Iso" decay trees in the Z simulation file, with W cuts. These histograms are statistically limited, as seen in Table \ref{tab: W events}.
    \end{enumerate}
    
    \item Retrieve $W$ isolated data: The function \texttt{make$\_$histogram} is called. $p_T$ is plotted for the Wp and Wm ``Iso" decay trees in the experimental data, with W cuts.
    
    \item Calculate Z fraction in W data:
    \begin{enumerate}
        \item Retrieve histograms for $m_Z$ distributions of the ``Z" tree in both data and the $Z$ simulation. The function \texttt{make$\_$histogram} is called for each, using Z cuts.
        \item Calculate the predicted number of $Z$ events in the data of $W^+$ and $W^-$ using
        \begin{equation}
            N_Z[{\rm predicted \; in \;} W {\rm\; data}] = N_Z[{\rm reconstructed \; as \;} W {\rm \; in \;} Z {\rm \; simulation}] \times \frac{N_Z[Z {\rm \; data}]}{N_Z[Z {\rm\; simulation}]},
        \end{equation}
        where $N_Z$ is the number of Z events, with the tree counted referred to in square brackets. The fraction of $Z$ in $W$ data, fraction$_Z$, is then calculated as
        \begin{equation}
            {\rm fraction}_Z = \frac{N_Z[{\rm predicted \; in \;} W {\rm\; data}]}{N_W[W {\rm \; data}]}.
        \end{equation}
        [$W$ data] refers to ``WpIso" and ``WmIso" trees, and [$Z$ data] to the ``Z" tree in the data file. \newline
        [reconstructed as $W$ in $Z$ simulation] refers to ``WpIso" and ``WmIso" trees, and [$Z$ simulation] to the ``Z" tree in the $Z$ simulation file. \newline 
        The number of events is counted in each of these trees by taking the integral of their histogram, each retrieved in the above steps. \newline
        The calculated $Z$ fractions are given in Section \ref{sec: W background results}.
        The statistical uncertainity is calculated using the relative uncertainty $\sqrt{N}/N$ on each count in fraction$_Z$.
        %The number of events in each of these trees is calculated by taking the integral of a histogram of the tree. For the ``W" trees, the $Z$ background and $W$ isolated data histograms discussed above are used, whereas for ``Z" trees, the invariant mass of $Z$ ``Z$\_$M" is plotted with the kinematic cuts used in Section \ref{sec: Z xsec}. 
    \end{enumerate}
    
    \item Run TFractionFitter on the background contributions to determine the fractions they constitute in the $W$ data. The function \texttt{fraction$\_$fitter} is called to do this. The previously determined $Z$ fraction is passed to this function, and is constrained to this value with its statistical uncertainity. \newline
    This fit produces the plots in Figure \ref{fig: W fraction fit}, and the fit fractions for each background contribution are given in \ref{sec: W background results}.
    
    \item Produce the fit model: the function \texttt{produce$\_$fit$\_$model} is called to do this. The histogram of each background contribution is scaled to the $W$ experimental data by
    \begin{equation}
        \frac{{\rm fraction} \times N_W[{\rm W \; data}]}{N[{\rm histogram}]}.
    \end{equation}
    The contributions of each background are summed to give the overall fit model, and this is plotted alongside the data and each individual background contribution, as shown in Figure \ref{fig: W fit model}.
\end{enumerate}




\subsubsection{Results of Background Analysis} \label{sec: W background results}
% include fit files, events counted in Z fraction calculation,

% Exponential fit of pion kaon background
\begin{figure*}[h]
  \centering
  \includegraphics[clip, trim = 0.6cm 0.1cm 2.1cm 1.5cm, width=0.49\textwidth]{W+_expo_back_plot.png}
  \includegraphics[clip, trim = 0.6cm 0.1cm 2.1cm 1.5cm, width=0.49\textwidth]{W-_expo_back_plot.png} \hspace{3mm}
  \vspace{-4mm}
  \caption{\small Exponential Fits of $p_T$ distributions for $W^+$ (left) and $W^-$ (right) in the ``SingleTrackNoIso" trees; high $p_T$ tracks without a muon identification requirement. This is mostly comprised of charged pions and kaons, forming the $\pi/K$ background.}
  \label{fig: W exp back fit}
\end{figure*}

% Table of event numbers
\input{results/W_event_table_output.tex}

% Z fractions
\textbf{The $Z$ fractions are calculated to be}: \newline
\input{results/Wp_Z_frac_output.tex}
\input{results/Wm_Z_frac_output.tex}

% Fit fractions
\textbf{The background fractions are determined by TFractionFitter to be:} \newline
\input{results/Wp_fit_fracs_output.tex}
\newline
\input{results/Wm_fit_fracs_output.tex}

% Fraction fitter plots
\begin{figure*}[h]
\centering
\includegraphics[clip, trim = 0.2cm 0.3cm 2.2cm 1.6cm, width=0.49\textwidth]{W+_fraction_fit.png}
\includegraphics[clip, trim = 0.2cm 0.3cm 2.2cm 1.6cm, width=0.49\textwidth]{W-_fraction_fit.png} 
\vspace{-4mm}
\caption{\small Plots of fits produced by TFractionFitter carried out on $W^+$ (left) and $W^-$ (right) data, with background fractions from $\pi/K \xrightarrow{} \mu\nu$, $W$ signal, and $Z \xrightarrow{} \mu\mu$. Crosses indicate the $W$ data, and the line indicates the fit. The $\chi^2/$degrees of freedom of the fit is given.}
\label{fig: W fraction fit}
\end{figure*}

% Fit model plots
\begin{figure*}[h]
\centering
\includegraphics[clip, trim = 0.2cm 0.3cm 2.2cm 1.6cm, width=0.49\textwidth]{W+_fit_model.png}
\includegraphics[clip, trim = 0.2cm 0.3cm 2.2cm 1.6cm, width=0.49\textwidth]{W-_fit_model.png} 
\vspace{-4mm}
\caption{\small Fit models for $W^+$ (left) and $W^+$ (right). The data are indicated with crosses, and the fit model calculated as the sum of all background contributions is plotted in black. Separate background contributions are plotted for $\pi/K \xrightarrow{} \mu\nu$ (blue), $W$ signal (red), and $Z \xrightarrow{} \mu\mu$ (green), with their fit fractions listed. The $\chi^2/$degrees of freedom of the fraction fit is given.}
\label{fig: W fit model}
\end{figure*}

\bibliography{refs}{}
\bibliographystyle{plain}

\end{document}
